\documentclass[12pt]{article}

\usepackage{sbc-template}

\usepackage{graphicx,url}
 
\usepackage[utf8]{inputenc}  
     
\sloppy

\title{TSO - 2024.2 - JOÃO PEDRO FERREIRA DUARTE}

\author{João Pedro F. Duarte\inst{1} }

\address{Bacharelado em Engenharia de Computação\\
Centro Federal de Educação Tecnológica de Minas Gerais (CEFET-MG)\\
   Minas Gerais -- MG -- Brasil
  \email{\ joaopedrofduarte@outlook.com}
}

\begin{document} 

\maketitle

\section{O Conto do Bonde Desgovernado} 

\begin{abstract}
  The text follows the style of argumentative dissertation, containing two pages and     four paragraphs that are divided in: Introduction, first case, second case and         conclusion. The text model used was made available in the official domain of SBC       (Sociedade Brasileira de Computação). The purpose of the text is to inform the         author's opinion on two situations cited in Michael J. Sandel's book, "Justice -       What it is to do a certain thing", and explain its reasons.
\end{abstract}
     
\begin{resumo} 
  O texto segue o estilo de dissertação argumentativa, contendo duas páginas e quatro   parágrafos que se dividem em introdução, primeiro caso, segundo caso e                 conclusão. O modelo de texto utilizado foi o disponibilizado no domínio               oficial  da SBC(Sociedade Brasileira de Computação). O objetivo do texto é             informar a  opinião do autor sobre duas situações citadas no livro “Justiça - o que   é fazer a coisa certa” de Michael J. Sandel, e explicar seus motivos.
\end{resumo}


\subsection{Introdução}

O texto a ser tem como objetivo demonstrar a minha opinião para com a história do   bonde desgovernado, na qual ele se encontra em duas situações e em cada uma delas   deve escolher entre duas alternativas, explicando motivos e razões das quais         resultaram suas escolhas.

\subsection{Primeiro Caso} \label{sec:firstpage}

Na primeira versão, dentro do contexto inserido pelo livro, ao estar guiando um bonde desgovernado deve se escolher entre atropelar cinco operários em uma linha ou atropelar um único operário em uma linha alternativa. Com base nos ensinamentos a mim repassados se houvesse a possibilidade de não matar ninguém, essa seria a minha escolha. Mas dentre as alternativas descritas acredito que pelo choque da situação eu não teria capacidade de escolher logicamente e ficaria paralisado. Porém assim como o algoritmo de uma Inteligência Artificial, que realiza escolhas de acordo com a proximidade da escolha que seria a “certa”, creio que a minha escolha seria desviar o bonde para a linha que continha apenas um operário, assumindo o risco de matar menos pessoas logo evitando assim uma tragédia maior e que seria a escolha menos errada e, além de haver a possibilidade de o operário escapar.

\subsection{Segundo Caso}

Na segunda versão, o livro insere um contexto no qual você se encontra em uma ponte acima dos trilhos do bonde, e deve escolher entre ver os cinco operários serem atropelados ou empurrar um homem corpulento da ponte sobre os trilhos com o objetivo de parar o bonde afim de salvar os cinco operários. Com base nos ensinamentos a mim repassados, considero essa segunda versão mais complicada, embora ela tenha uma essência parecida com a primeira, a de escolher entre atropelar cinco operários ou apenas um, ela envolve uma terceira pessoa neutra, mesmo com a ideia de empurrar uma pessoa para a morte sendo macabra, creio que ainda assim escolheria ela, pois os cinco operários teriam uma chance maior de escapar do atropelamento.

\subsection{Conclusão}

Essa pequena história nos mostra que por mais que tentemos escolher o que é considerado o certo, sempre irão ocorrer situações que nos posicionam fora da nossa zona de conforto, colocando a prova nossos valores morais, nos forçando fazer escolhas que são necessárias e que irão definir nosso futuro, nos mostrando que nem sempre vamos poder escolher fazer o certo, mas ter de escolher fazer o necessário, por mais difícil que seja. Várias situações parecidas são postas a nós todos os dias, porém com diferentes magnitudes, o modo como nos portamos diante dessas decisões é o que nos define.

\subsection{Referências}

\cite{sandel:2015}.

\bibliographystyle{sbc}
\bibliography{sbc-template}

\newpage

\section{Delineando o perfil dos alunos egressos do BCC do IFCE - Campus Aracati.}


\subsection{Introdução}


\newpage
\section{Trabalho}






\end{document}
