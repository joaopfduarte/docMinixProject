\documentclass[12pt]{article}

\usepackage{sbc-template}

\usepackage{graphicx,url}
 
\usepackage[utf8]{inputenc}
\usepackage{holtxdoc}
\usepackage{hyperref}
\usepackage{lstdoc}

\hypersetup{
    colorlinks=true,      % ativa as cores dos links
    linkcolor=black,       % cor dos links internos
    urlcolor=black,        % cor dos links externos
    citecolor=black        % cor das citações
}

     
\sloppy

\title{TSO - 2024.2 - JOÃO PEDRO FERREIRA DUARTE}

\author{João Pedro F. Duarte\inst{1} }

\address{Bacharelado em Engenharia de Computação\\
Centro Federal de Educação Tecnológica de Minas Gerais (CEFET-MG)\\
   Minas Gerais -- MG -- Brasil
  \email{\ joaopedrofduarte@outlook.com}
}

\begin{document} 

\maketitle

\section{Implementação da Syscall no SO Minix3}

\begin{abstract}
    In Operating Systems, system calls, \texttt{Syscalls}, are functions renewed to perform some type of specific service in favor of the processes carried out by the OS.
    With this same objective, the present study presents the implementation of a \texttt{Syscall}, which we will call \textbf{useless}, within the Minix3.2.1 Process Manager Server, for purposes
    study on this process.
\end{abstract}
     
\begin{resumo}
Em Sistemas Operacionais, chamadas de sistema, \texttt{Syscalls}, são funções implementadas para realizar algum tipo de serviço específico em favor dos processos realizados pelo SO.
    Com este mesmo objetivo, o presente estudo apresenta a implementação de uma \texttt{Syscall}, que a chamaremos de \textbf{inútil}, dentro do Servidor Gerenciador de Processo do Minix3.2.1, para fins
de estudo sobre este processo.
\end{resumo}


\subsection{Introdução} %Configuracoes e ambiente

O presente texto, com objetivo de instriução, serve para que qualquer programados com o intuito de estudar mais sobre os conceitos relacionados a Sistemas Operacionais
, através do uso do Minix, em sua versão $3.2.1$, possa realizar todos os passos necessários para implementar, desde o início, a criação de uma chamada de sistema. Em
maiores detalhes, temos aqui as referências \cite{creating_syscall} e \cite{tanenbaum_modern_os} como fontes base para entendimento do Kernel do sistema operacional em uso,
como também da documentação presente no site do Minix para desenvolvedores conforme \cite{minix3docs}. \href{https://www.minix3.org/doc/}{Link da documentação}.
%

Sobre a ambientação necessária para o desenvolvimento deste projeto, o programador deve realizar a instalação de uma máquina virtual minix, na versão correta para este projeto
e realizar todas as etapas de instalação do sistema operacional conforme documentação oficial supracitada. O link referente à plataforma onde deve ser criada a máquina virtual,
como também ao local onde estará a versão correta do sistema operacional e a parte da documentação onde constam as etapas de instalação do sistema, podem ser consultados a seguir:
\href{https://www.virtualbox.org/wiki/Downloads}{[Virtual Box]}, \href{https://wiki.minix3.org/doku.php?id=www:download:start}{[Minix versão 3.2.1]} e \href{https://wiki.minix3.org/doku.php?id=usersguide:runningonvirtualbox}{[processo de instalação]}.


\subsection{Primeiro caso} \label{sec:1sec}
Inicialmente, para dar início ao processo de criação da chamada de sistema, tem-se como focos de alteração e estudo para fins de entendimento do processo interno do
Kernel do sistema operacional, a aplicação das seguintes rotas para atualização de dados e manipulação.

\begin{enumerate}
    \item /callnr.h: Aqui está a listagem de todas as chamadas de sistema presentes no Minix.
    \item /table.c: Aqui estão os nomes de cada uma das rotinas das chamadas de sistema do Minix.
    \item /proto.h: Protótipo de todas as rotinas de chamadas de sistema do sistema operacinal.
    \item /misc.c: Chamada de sistema de todas as Syscalls do tipo \texttt{Misc}.
    \item /minhachamadalib.h: Este é o "include file" que realiza a chamada da chamada de sistema \texttt{inútil}.
    \item /teste1.c: Arquivo de teste criado para realizar o teste da chamada \texttt{inútil}.
\end{enumerate}


\subsection{Segundo Caso} \label{sec:2sec}

Na segunda versão, o livro insere um contexto no qual você se encontra em uma ponte acima dos trilhos do bonde, e deve escolher entre ver os cinco operários serem atropelados ou empurrar um homem corpulento da ponte sobre os trilhos com o objetivo de parar o bonde afim de salvar os cinco operários. Com base nos ensinamentos a mim repassados, considero essa segunda versão mais complicada, embora ela tenha uma essência parecida com a primeira, a de escolher entre atropelar cinco operários ou apenas um, ela envolve uma terceira pessoa neutra, mesmo com a ideia de empurrar uma pessoa para a morte sendo macabra, creio que ainda assim escolheria ela, pois os cinco operários teriam uma chance maior de escapar do atropelamento.

\subsection{Terceiro Caso} \label{sec:3sec}
\subsection{Quarto Caso} \label{sec:4sec}
\subsection{Quinto Caso} \label{sec:5sec}
\subsection{Sexto Caso} \label{sec:6sec}
\subsection{Sétimo Caso} \label{sec:7sec}

\subsection{Conclusão}

Essa pequena história nos mostra que por mais que tentemos escolher o que é considerado o certo, sempre irão ocorrer situações que nos posicionam fora da nossa zona de conforto, colocando a prova nossos valores morais, nos forçando fazer escolhas que são necessárias e que irão definir nosso futuro, nos mostrando que nem sempre vamos poder escolher fazer o certo, mas ter de escolher fazer o necessário, por mais difícil que seja. Várias situações parecidas são postas a nós todos os dias, porém com diferentes magnitudes, o modo como nos portamos diante dessas decisões é o que nos define.

\subsection{Referências}


\bibliographystyle{sbc}

\newpage
\bibliography{sbc-template}



\end{document}
